\documentclass{article}

\usepackage[a4paper,margin=3cm,headsep=1cm]{geometry}
\usepackage[utf8]{inputenc}

\usepackage{%
    booktabs,
    fancyhdr,
    pgfgantt,
    titlesec,
    clrscode3e,
    exercise,
}

\pagestyle{fancy}

\rhead{Algorithms and Data Structures, F25}
\lhead{Exercises part 1, Lecture 01}
\rfoot{Aalborg University, Copenhagen}
\lfoot{Andreas Holck Høeg-Petersen}

\begin{document}
\thispagestyle{fancy}

\begin{Exercise}
    \noindent
    Consider the pseudo-code for \proc{Find-Element}:

    \begin{codebox}
        \Procname{$\proc{Find-Element}(A, a)$}
        \li $j \gets 0$
        \li \For $i \gets 1$ \To \attrib{A}{length} \Do
            \li \If $A[i] \gets a$ \Then
                \li $j \gets i$
            \End
        \End
        \li \Return $j$
    \end{codebox}

    \Question
    Step \textit{meticulously} through the algorithm for each of the following
    inputs. Note down each line that is executed, what the value of $i$ and $j$
    is and what is returned in the end.

    \begin{itemize}
        \item (`m', `f', `a', `b', `k'), `b'
        \item (7, 2, 1, 4), 7
        \item (), `d'
        \item (0, 1, 0, 1, 0, 1, 0), 1
        \item (3, 4, 2, 4), 4
        \item (`p', `x', `f', `l'), `m'
    \end{itemize}

    \medskip
    \ExeText
    Consider the pseudo-code for \proc{Find-Element-v2}, an alternative version
    the algorithm:
    
    \begin{codebox}
        \Procname{$\proc{Find-Element-v2}(A, a)$}
        \li $j \gets \attrib{A}{length}$
        \li \While $i > 0$ \Do
            \li \If $A[i] = a$ \Then
                \li $j \gets i$
            \End
            \li $i = i - 1$
        \End
        \li \Return $j$
    \end{codebox}

    \Question
    Does \proc{Find-Element} and \proc{Find-Element-v2} solve the same problem?
    Step through the algorithm to convince yourself!

    \medskip
    \ExeText
    Consider the pseudo-code for \proc{Find-Element-v3}, an alternative version
    the algorithm:
    
    \begin{codebox}
        \Procname{$\proc{Find-Element-v3}(A, a)$}
        \li $i \gets \attrib{A}{length}$
        \li \While $i > 0$ and $A[i] \neq a$ \Then
            \li $i = i - 1$
        \End
        \li \Return $i$
    \end{codebox}

    \Question
    Does \proc{Find-Element-v3} solve the same problem as either or both of the
    two previous algorithms? Once again, journey through the tedious process of
    stepping through the algorithm to convince yourself!

\end{Exercise}
    
\begin{Exercise}

    \Question
    Write pseudo-code for an algorithm \proc{Count-Instances}, which takes as
    input a sequence $A$ of characters and a character $a$ and outputs the
    number of occurrences of $a$ in $A$. So, for the input pair ((`a', `b', `b',
    `a'), `a') the output should be 2.
    
    \medskip
    \Question
    Write pseudo-code for an algorithm \proc{Check-Equality}, which takes as
    input two sequences $A$ and $B$ of characters and outputs \const{True} if
    $A$ and $B$ are identical and \const{False} otherwise. NB: Note that it is
    not required that the two sequences are of the same length!
    
\end{Exercise}


\end{document}
