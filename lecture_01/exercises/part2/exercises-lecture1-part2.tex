\documentclass{article}

\usepackage[a4paper,margin=3cm,headsep=1cm]{geometry}
\usepackage[utf8]{inputenc}

\usepackage{%
    booktabs,
    fancyhdr,
    pgfgantt,
    titlesec,
    clrscode3e,
    exercise,
}

\pagestyle{fancy}

\rhead{Algorithms and Data Structures, S25}
\lhead{Exercises part 2, Lecture 01}
\rfoot{Aalborg University, Copenhagen}
\lfoot{Andreas Holck Høeg-Petersen}

\begin{document}
\thispagestyle{fancy}


\begin{Exercise}
    \noindent
    As a warm-up for analyzing runtime, consider the pseudo-code for
    \proc{Sum-Even-Numbers} which takes a sequence of integers and returns the
    sum of all even numbers:

    \begin{codebox}
        \Procname{$\proc{Sum-Even-Numbers}(A)$}
        \li $\key{sum} \gets 0$
        \li \For $i \gets 1$ \To \attrib{A}{length} \Do
            \li \If $A[i] \mod{2} \isequal 0$ \Then
                \li $\key{sum} \gets \key{sum} + A[i]$
            \End
        \End
        \li \Return $\key{sum}$
    \end{codebox}

    \noindent
    Assume that each line $i$ takes $c_i$ time to execute.

    \Question
    What is the \textit{exact} running time $T(n)$ of
    \proc{Sum-Even-Numbers}?

    \Question
    What is the \textit{worst case} running time and what input would cause
    this?

\end{Exercise}

\medskip

\begin{Exercise}
    \noindent
    Consider the problem of finding the two smallest numbers in a non-empty
    sequence of numbers, not necessarily sorted.

    \Question
    Formalise the above as a computational problem (be careful to precisely
    define the input, the output and their relationship).

    \Question
    Write the pseudo-code of an algorithm that solves the above computational
    problem, assuming the sequence is given as an array $A[1\ldots n]$.

    \Question
    Assume that line $i$ of your pseudo-code takes constant time $c_i$ to
    execute. What is the worst case running time of your algorithm?
\end{Exercise}

\medskip

\begin{Exercise}
    \ExeText
    Consider the problem of reversing an array of numbers, ie.\ so the last
    element becomes the first, the second last becomes the second, etc.\ For
    example, the reverse of $[1,2,3,4,5]$ is $[5,4,3,2,1]$.

    \Question
    Write the pseudo-code for an algorithm that solves this problem.
    \textit{Hint:} In pseudo-code, you can just create a new array of size $n$
    by writing `$B$ = new array of size $n$'

    \Question
    What is the worst case running time of your algorithm?

    \ExeText
    We measure space almost like we measure time, except instead of assuming a
    constant cost of each line, we assume a constant cost $c$ of each new
    variable (since the value associated with the variable is stored in memory).
    An array of size $n$ then takes $c \cdot n$ space, since we are essentially
    creating $n$ variables, one for each slot in the array.

    \Question
    Assuming you created another array for the reversed sequence in the above
    question, try now to propose an algorithm that only uses a contant amount of
    extra space. What is the worst case running time of your algorithm?
\end{Exercise}

\begin{Exercise}
    \noindent
    Consider the following star-printing algorithms:

    \begin{minipage}[t]{.3\textwidth}
        \begin{codebox}
            \Procname{$\proc{Print-Logarithmic}(n)$}
            \li $i = n$
            \li \While $i \geq 1$ \Do
                \li print *
                \li $i = i / 2$
            \End
        \end{codebox}
    \end{minipage}%
    \hfill
    \begin{minipage}[t]{.3\textwidth}
        \begin{codebox}
            \Procname{$\proc{Print-Linear}(n)$}
            \li \For $i = 1$ \To $n$ \Do
                \li print *
            \End
        \end{codebox}
    \end{minipage}%
    \hfill
    \begin{minipage}[t]{.3\textwidth}
        \begin{codebox}
            \Procname{$\proc{Print-Squared}(n)$}
            \li \For $i=1$ \To $n$ \Do
                \li \For $j=1$ \To $n$ \Do
                    \li print *
                \End
            \End
        \end{codebox}
    \end{minipage}

    \Question
    How many stars are printed with by each algorithm, when you call
    \proc{Print-Logarithmic}($n$) with $n = 32, 16, 8, 4, 2$ and
    \proc{Print-Linear}($n$) and \proc{Print-Squared}($n$) with $n = 1, 2, 3, 4,
    5$?

    \Question
    Is there a relation between the number of stars printed and the running time
    of the algorithms?
\end{Exercise}



\end{document}
