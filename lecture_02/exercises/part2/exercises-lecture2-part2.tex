\documentclass{article}

\usepackage[a4paper,margin=3cm,headsep=1cm]{geometry}
\usepackage[utf8]{inputenc}

\usepackage{%
    booktabs,
    fancyhdr,
    pgfgantt,
    titlesec,
    clrscode3e,
    exercise,
}

\pagestyle{fancy}

\rhead{Algorithms and Data Structures, S25}
\lhead{Exercises part 2, Lecture 02}
\rfoot{Aalborg University, Copenhagen}
\lfoot{Andreas Holck Høeg-Petersen}

\begin{document}
\thispagestyle{fancy}

\begin{Exercise}

    \noindent
    Give the asymptotic tight bounds ($\Theta)$ on the following functions of
    $n$. Here, $k \geq 1$ and $c > 1$ are constants. Hint: you might want to
    take a look at some of the classic functions and their properties in Section
    3.3 CLRS.

    \Question
    $0.001n^2 + 70000n$
    
    \Question
    $2^n + n^{1000}$

    \Question
    $n^k + c^n$

    \Question
    $20 \log n + n^k$

    \Question
    $2^n + 2^{n/2}$

    \Question
    $n^{\log c} + c^{\log n}$

    \Question
    $13n^4 + 7n^3 - 9n^2 + 127n + \frac{n!}{7}$
\end{Exercise}

\begin{Exercise}
    
    \noindent
    Consider the following function $f(n) = 3n \log n + 15n + 1800$ and prove
    that it is in $O(n \log n)$ (hint: find constants $c$ and $n_0$ such that
    $f(n) \leq c(n \log n)$ for all $n \geq n_0$.

\end{Exercise}

\begin{Exercise}

    \noindent
    Consider the pseudo-code for the following mysterious  sorting function
    \proc{Sort($A$)}:
    
    \begin{codebox}
        \Procname{$\proc{Sort}(A)$}
        \li $n \gets \attrib{A}{length}$
        \li \For $i \gets 1$ \To $n$ \Do
            \li \For $j \gets i+1$ \To $n$ \Do
                \li \If $A[i] > A[j]$ \Then
                    \li $\id{key} \gets A[i]$
                    \li $A[i] \gets A[j]$
                    \li $A[j] \gets \id{key}$
                \End
            \End
        \End
        \li \Return $A$
    \end{codebox}

    \Question
    Explain informally how the algorithm works.

    \Question
    Prove that \proc{Sort} correctly solves the solving problem (hint: determine
    suitable invariants for both loops)

    \Question
    Determine the asymptotic worst-case running time in terms of upper bound $O$
    and lower bound $\Omega$.

\end{Exercise}

\end{document}

