\documentclass{article}

\usepackage[a4paper,margin=3cm,headsep=1cm]{geometry}
\usepackage[utf8]{inputenc}
\usepackage{%
    booktabs,
    fancyhdr,
    pgfgantt,
    titlesec,
    clrscode3e,
    exercise,
}

\pagestyle{fancy}

\rhead{Algorithms and Data Structures, S25}
\lhead{Exercises part 1, Lecture 03}
\rfoot{Aalborg University, Copenhagen}
\lfoot{Andreas Holck Høeg-Petersen}

\begin{document}
\thispagestyle{fancy}

\begin{Exercise}

    \Question
    Run the \proc{Merge-Sort} algorithm on the following array of characters $A
    = \langle c, w, x, m, o, z, q, e \rangle$. Give the state of the array after
    five calls to \proc{Merge} have been performed.

    \Question
    State a loop invariant for the \textbf{while} loop of lines 12-18 of the
    \proc{Merge} procedure (from the book). Show how to use it along with the
    \textbf{while} loops of lines 20-23 and 24-27, to prove that the
    \proc{Merge} procedure is correct. (CLRS 2.3-3)

    \Question
    You can also think of insertion sort as a recursive algorithm. In order to
    sort $A[1:n]$, recursively sort the subarray $A[1:n-1]$ and then insert
    $A[n]$ into the sorted subarray $A[1:n-1]$. Write pseudo-code for thise
    recursive version of insertion sort. Give a recursion for its worst-case
    running time (might be easier after the second part of the lecture). (CLRS
    2.3-5)

\end{Exercise}

\begin{Exercise}
    
    \Question
    Using Figure 7.1 as a model, illustrate the operation of \proc{Partition} on
    the array $A = \langle 13, 19, 9, 5, 12, 8, 7, 4, 21, 2, 6, 11 \rangle$.
    (CLRS 7.1-1)

    \Question
    Give a brief argument that the running time of \proc{Partition} on a
    subarray of size $n$ is $\Theta(n)$. (CLRS 7.1-3)

    \Question
    Show that the running time of \proc{Quicksort} is $\Theta(n^2)$ when the
    array $A$ contains distinct elements and is sorted in decreasing order.
    (CLRS 7.2-3)

\end{Exercise}

\begin{Exercise}
    \ExeText Consider the problem of finding the smallest element in a nonempty
    array of numbers $A[1:n]$.

    \Question
    Write an \textit{incremental} algorithm that solves the above problem and
    determine its asymptotic worst-case running time.

    \Question
    Write an \textit{divide-and-conquer} algorithm that solves the above problem
    and determine its asymptotic worst-case running time.

    \Question
    Assume that the length of $A$ is a power of 2. Write a recurrence describing
    how many comparison operations (among elements of $A$) your
    divide-and-conquer algorithm performs (might be easier after the second part
    of the lecture).
\end{Exercise}

\end{document}
