\documentclass{article}

\usepackage[a4paper,margin=3cm,headsep=1cm]{geometry}
\usepackage[utf8]{inputenc}
\usepackage{%
    booktabs,
    fancyhdr,
    pgfgantt,
    titlesec,
    clrscode3e,
    exercise,
}

\pagestyle{fancy}

\rhead{Algorithms and Data Structures, S25}
\lhead{Exercises part 2, Lecture 03}
\rfoot{Aalborg University, Copenhagen}
\lfoot{Andreas Holck Høeg-Petersen}

\begin{document}
\thispagestyle{fancy}

\begin{Exercise}

    \ExeText
    Give asymptotic upper and lower bounds for $T(n)$ in each of the following
    algorithmic recurrences. Justify your answer by identifying $a$, $b$ and
    $f(n)$ and compare the watershed function $n^{\log_b a}$ with the driving
    function $f(n)$.

    \Question
    $T(n) = 2T(n/4) + 1$

    \Question
    $T(n) = 2T(n/4) + \sqrt{n}$

    \Question
    $T(n) = 2T(n/4) + n$

    \Question
    $T(n) = 2T(n/2) + n^3$

    \Question
    $T(n) = T(8n/11) + n$

    \Question
    $T(n) = 16T(n/4) + n^2$

    \Question
    $T(n) = 4T(n/2) + n^2 \log n$

    \Question
    $T(n) = 8T(n/3) + n^2$

    \Question
    $T(n) = 7T(n/2) + n^2 \log n$

\end{Exercise}

\begin{Exercise}
    
    \ExeText
    \noindent
    Finish the exercises from the first session. If you are done or want to train
    your creative problem solving consider the following problem: \\

    \noindent
    You have $n$ nuts and $n$ bolts. Each nut goes together with one bolt, but
    you have forgot which ones belong together! And you are in a hurry, so the
    brute force solution of trying every nut on every bolt until you find a match
    does not work for you (what's the runtime of this approach?). Luckily, you
    know about \proc{Quicksort} and you have an idea that this might help you to
    solve the problem in $\Theta{n \log n}$. \\

    \noindent
    In this situation, you cannot compare two nuts or two bolts - you can only
    check if a bolt is to big or small to fit in a nut (or vice versa). You have
    solved the problem, if you have arranged both nuts and bolts in two
    sequences so the $i$th nut fits the $i$th bolt for all $i \in 1\ldots n$. \\

    \noindent
    \textit{Hint:} Maybe you can use the bolts to partition the nuts and the
    nuts to partition the bolts? Remember, that doing more than one partioning
    operation per recursive call does not necessarily change the asymptotic
    runtime.
\end{Exercise}

\end{document}
