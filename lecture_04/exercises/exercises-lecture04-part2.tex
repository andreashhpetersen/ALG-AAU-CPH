\documentclass{article}

\usepackage[a4paper,margin=3cm,headsep=1cm]{geometry}
\usepackage[utf8]{inputenc}
\usepackage{%
    booktabs,
    fancyhdr,
    pgfgantt,
    titlesec,
    clrscode3e,
    exercise,
    tikz,
    subcaption,
    csquotes
}
\usetikzlibrary{trees}

\pagestyle{fancy}

\rhead{Algorithms and Data Structures, S25}
\lhead{Exercises part 2, Lecture 04}
\rfoot{Aalborg University, Copenhagen}
\lfoot{Andreas Holck Høeg-Petersen}

\begin{document}
\thispagestyle{fancy}

\begin{Exercise}[title={Typing some traces}]

    \ExeText
    \noindent
    These exercises are simply meant to give you some first-hand experience with
    the algorithms we have seen today on heaps. The next exercises are more fun
    and creative.

    \Question
    Using Figure 6.2 as a model, illustrate the operation of
    $\proc{Max-Heapify}(A,3)$ on the array $A = \langle 27, 17, 3, 16, 10, 1, 5,
    7, 12, 4, 8, 9, 0 \rangle$. (CLRS 6.2-1)

    \Question
    Using Figure 6.3 as a model, illustrate the operation of
    $\proc{Build-Max-Heap}$ on the array $A = \langle 5, 3, 17, 10, 84, 19, 6,
    22, 9 \rangle$. (CLRS 6.3-1)

    \Question
    Using Figure 6.4 as a model, illustrate the operation of $\proc{Heapsort}$
    on the array $A = \langle 5, 13, 2, 25, 7, 17, 20, 8, 4 \rangle$. (CLRS
    6.4-1)

    \Question
    Suppose that the objects in a max-priority queue are just keys (ie.\ no need
    to do $\attrib{A[i]}{key})$. Illustrate the operation of
    $\proc{Max-Heap-Insert}(A,10)$ on the heap $A = \langle 15, 13, 9, 5, 12, 8,
    7, 4, 0, 6, 2, 1 \rangle$. (NB: just use the $\proc{Insert}$ operation
    presented in the slides)

\end{Exercise}

\begin{Exercise}[title={Exam question}]
    \noindent
    This is a sub-question from the 2024 exam set. \\

    \ExeText
    \noindent
    `Game of Thrones' is a pretty great show and you are a huge fan.  It only
    has two problems: the last 2 (or 4, to be honest) seasons and the fact that
    not \textit{every} episode contains dragons. You really love dragons.
    Therefore, you decide to watch every episode and note down how many minutes
    of dragon action each episode has. After each episode you want to add the
    episode to some data structure, that will you let easily select and rewatch
    the episode(s) with the most amount of dragon. \\

    \Question
    What data structure might fit your need --- a stack, a queue, a linked list
    or a priority queue? Remember that you don't want to sort all the episodes
    each time you add a new one or want to get the one with the most dragon.
    Explain your answer.

\end{Exercise}

\begin{Exercise}[title={Loop invariant for $\proc{Heapsort}$}]
    \noindent
    Argue the correctness of \proc{Heapsort} using the following loop invariant
    (CLRS 6.4-2):

    \begin{displayquote}
        At the start of each iteration of the \textbf{for} loop of lines 2-5,
        the subarray $A[1:i]$ is a max-heap containing the $i$ smallest elements
        of $A[1:n]$, and the subarray $A[i+1:n]$ contains the $n-i$ largest
        elements of $A[1:n]$, sorted.
    \end{displayquote}
\end{Exercise}

\begin{Exercise}[title={Priority Queues}]
    \noindent
    These questions concern priority queues.

    \Question
    The largest element in a heap must appear in position 1, and the second
    largest must be in position 2 or position 3. Give the list of positions in a
    heap of size 31 where the $k$th largest \textit{can} appear and
    \textit{cannot} appear for $k=2,3,4$ (assuming the values are distinct).

    \Question
    Explain how to use a priority queue to implement the stack and queue
    datastructures.

    \Question
    The operation $\proc{Max-Heap-Delete}(A,x)$ deletes the object $x$ from
    max-heap $A$. Give a pseudo-code implementation of $\proc{Max-Heap-Delete}$
    for an $n$-element max-heap that runs in $O(\log n)$ time. (CLRS 6.5-10)

\end{Exercise}

\end{document}

