\documentclass{article}

\usepackage[a4paper,margin=3cm,headsep=1cm]{geometry}
\usepackage[utf8]{inputenc}
\usepackage{%
    booktabs,
    fancyhdr,
    pgfgantt,
    titlesec,
    clrscode3e,
    exercise,
    tikz,
    subcaption
}
\usetikzlibrary{trees}

\pagestyle{fancy}

\rhead{Algorithms and Data Structures, S25}
\lhead{Exercises part 1, Lecture 05}
\rfoot{Aalborg University, Copenhagen}
\lfoot{Andreas Holck Høeg-Petersen}

\begin{document}
\thispagestyle{fancy}

\begin{Exercise}[title={Simple training exercises}]

    \Question
    For the keys $\{ 1,4,5,10,16,17,21 \}$, draw binary search trees of heights
    2, 3, 4, 5 and 6 (CLRS 12.1-1)

    \Question
    You are searching for the number 363 in binary search tree containing
    numbers between 1 and 1000. Which of the following sequences \textit{cannot}
    be the sequence of nodes examined? (CLRS 12.2-1)

    \begin{itemize}
        \item 2, 252, 401, 398, 330, 344, 397, 363
        \item 924, 220, 911, 244, 898, 258, 362, 363
        \item 925, 202, 911, 240, 912, 245, 363
        \item 2, 399, 387, 219, 266, 382, 381, 278, 363
        \item 935, 278, 347, 621, 299, 392, 358, 363
    \end{itemize}

    \Question
    Give the sequence of nodes examined when the following calls are made on the
    binary search tree given beneath:

    \begin{minipage}{.35\textwidth}
    \begin{itemize}
        \item $\proc{Tree-Search}(T,\textrm{M})$
        \item $\proc{Tree-Minimum}(\textrm{Q})$
        \item $\proc{Tree-Successor}(\textrm{M})$
    \end{itemize}
    \end{minipage}
    \hfill
    \begin{minipage}{.6\textwidth}
        \centering
        
        \begin{tikzpicture}[
            level 1/.style={sibling distance=10em},
            level 2/.style={sibling distance=7em},
            level 3/.style={sibling distance=5em},
        ]
        \node {E}
            child {node {D}
                child {node {A}}
                child {node[draw=none]{} }
            }
            child {node {Q}
                child {node {J}
                    child {node[draw=none]{}}
                    child {node {M}}
                }
                child {node {T}
                    child {node {S}}
                    child {node[draw=none]{}}
                }
            };
    \end{tikzpicture}

    \end{minipage}

    \Question
    Write the $\proc{Tree-Predecessor}$ procedure (CLRS 12.2-3)

\end{Exercise}

\begin{Exercise}[title={Fun creative exercises!}]
    \Question
    Professor Kilmer claims to have discovered a remarkable property of binary
    search trees.  Suppose that the search for key $k$ in a binary search tree
    ends up at a leaf.  Consider three sets: $A$, the keys to the left of the
    search path; $B$, the keys on the search path; and $C$, the keys to the
    right of the search path. Professor Kilmer claims that any three keys $a \in
    A$, $b \in B$, and $c \in  C$ must satisfy $a \leq b \leq c$. Give a
    smallest possible counterexample to the professor's claim. (CLRS 12.2-4)

    \Question
    Show that if a node in a binary search tree has two children, then its
    successor has no left child and its predecessor has no right child. (CLRS
    12.2-5)

    \Question
    An alternative method of performing an inorder tree walk of an $n$-node
    binary search tree finds the minimum element in the tree by calling
    $\proc{Tree-Minumum}$ and then making $n - 1$ calls to
    $\proc{Tree-Successor}$. Prove that this algorithm runs in $\Theta(n)$ time.
    (CLRS 12.2-7)

\end{Exercise}


\end{document}

