\documentclass{article}

\usepackage[a4paper,margin=3cm,headsep=1cm]{geometry}
\usepackage[utf8]{inputenc}
\usepackage{%
    booktabs,
    fancyhdr,
    pgfgantt,
    titlesec,
    clrscode3e,
    exercise,
    tikz,
    subcaption,
    csquotes
}
\usetikzlibrary{trees}

\pagestyle{fancy}

\rhead{Algorithms and Data Structures, S25}
\lhead{Exercises part 2, Lecture 05}
\rfoot{Aalborg University, Copenhagen}
\lfoot{Andreas Holck Høeg-Petersen}

\begin{document}
\thispagestyle{fancy}

\begin{Exercise}[title={Simple training exercises}]

    \Question
    Insert the keys `E', `A', `S', `Y', `Q', `U', `E', `S', `T', `I', `O', `N'
    in an initally empty BST $T$, and show the tree after inserting `Q' and
    after inserting `N'

    \Question
    Draw the sequence of BSTs that results when you delete the keys from the
    tree of the previous exercise in the order that the keys were inserted.

    \Question
    When $\proc{Tree-Delete}$ calls $\proc{Transplant}$, under what
    circumstances can the parameter $v$ of $\proc{Transplant}$ be $\const{NIL}$?
    (CLRS 12.3-4)
\end{Exercise}

\begin{Exercise}[title={Fun creative exercises!}]
    \Question
    Give a recursive version of the $\proc{Tree-Insert}$ procedure. (CLRS
    12.3-1)

    \Question
    You can sort a given set of $n$ numbers by first building a binary search
    tree containing these numbers (using $\proc{Tree-Insert}$ repeatedly to
    insert the numbers one by one) and then printing the numbers by an inorder
    tree walk. What are the worst- case and best-case running times for this
    sorting algorithm? (CLRS 12.3-3)

    \Question
    Suppose that instead of each node $x$ keeping the attribute $\attrib{x}{p}$,
    pointing to $x$’s parent, it keeps $\attrib{x}{succ}$, pointing to $x$’s
    successor. Give pseudocode for $\proc{Tree-Search}$, $\proc{Tree-Insert}$,
    and $\proc{Tree-Delete}$ on a binary search tree $T$ using this
    representation. These procedures should operate in $O(h)$ time, where $h$ is
    the height of the tree $T$. You may assume that all keys in the binary
    search tree are distinct. (Hint: You might wish to implement a subroutine
    that returns the parent of a node) (CLRS 12.3-6)
\end{Exercise}

\begin{Exercise}[title={Question from the 2024 re-exam}]
    
    \ExeText
    \noindent
    Consider the following algorithm. The input is a \textit{balanced}
    binary search tree $T$ with $n$ nodes and an array $A$ with $n$ keys,
    that \textit{may or may not} be in $T$. The algorithm uses the
    procedures \proc{Tree-Search} and \proc{Tree-Successor}
    from CLRS (Chapter 12.2). Assume zero-indexing.

    \begin{codebox}
        \Procname{$\proc{GetSuccessors}(T, A)$}
        \li $i = 0$
        \li $B$ = a new array of size $A.\mathit{length}$
            \li \For $k \in A$ \Do
            \li $x =$ \proc{Tree-Search}($T.\mathit{root}, k)$
                \li \If $x \neq \mathrm{Nil}$ \Then
                    \li $s =$ \proc{Tree-Successor}$(x)$
                    \li \If $s \neq \mathrm{Nil}$ \Then
                        \li $B[i]$ = $s$
                        \li $i = i + 1$
                    \End
                \End
            \End
        \li \Return $B$
    \end{codebox}


    \Question
    Give the asymptotic running time of \proc{GetSuccessors} \textit{in terms of
    n}. Argue for your answer, give as tight a bound as possible and remember to
    reduce your answer (ie.\ if you analyze the running time of each line, you
    should still provide an answer like $O(1)$ or $\Theta(2^n)$ for the complete
    algorithm).

    \Question
    The output of \proc{GetSuccessors} is an array $B$ of size $n$, but multiple
    slots in the array may be empty. Can you suggest a better data structure for
    $B$, that wouldn't have unused space? Argue for your suggestion.


\end{Exercise}

\end{document}

