\documentclass{article}

\usepackage[a4paper,margin=3cm,headsep=1cm]{geometry}
\usepackage[utf8]{inputenc}
\usepackage{%
    booktabs,
    fancyhdr,
    pgfgantt,
    titlesec,
    clrscode3e,
    exercise,
    tikz,
    subcaption
}
\usetikzlibrary{trees}

\pagestyle{fancy}

\rhead{Algorithms and Data Structures, S25}
\lhead{Exercises part 1, Lecture 06}
\rfoot{Aalborg University, Copenhagen}
\lfoot{Andreas Holck Høeg-Petersen}

\begin{document}
\thispagestyle{fancy}

\begin{Exercise}[title={Simple training exercises}]

    \Question
    Draw the red-black tree that results after $\proc{Tree-Insert}$ (ie.\ the
    insert procedure for normal binary search trees) is called on the tree in
    Figure 13.1 with key 36. If the inserted node is colored red, is the
    resulting tree a red-black tree? What if it is colored black? (CLRS 13.1-2)

    \Question
    In Figure 13.3 (after the rotation), perform a $\proc{Left-Rotate}$ on the
    node with key 11, then a $\proc{Right-Rotate}$ on the node with key 14 and
    then a $\proc{Left-Rotate}$ on the node with key 7.

    \Question
    Show the red-black trees that result after successively inserting the keys
    41, 38, 31, 12, 19, 8 into an initially empty red-black tree. (CLRS 13.3-2)

    \Question
    Suppose that the black-height of each of the subtrees $\alpha$, $\beta$,
    $\gamma$, $\delta$, $\epsilon$ in Figures 13.5 and 13.6 is $k$. Label each
    node each figure with its black-height to verify that the indicated
    transformation preserves property 5. (CLRS 13.3-3)
\end{Exercise}

\begin{Exercise}[title={Fun creative exercises!}]
    \Question
    Line 16 of $\proc{RB-Insert}$ sets the color of the newly inserted node z to
    red. If instead $z$'s color were set to black, then property 4 of a
    red-black tree would not be violated. Why not set $z$'s color to black?
    (CLRS 13.3-1)

    \Question
    Consider a red-black tree formed by inserting $n$ nodes with
    $\proc{RB-Insert}$. Argue that if $n > 1$, the tree has at least one red
    node. (CLRS 13.3-5)

    \bigskip
    \ExeText
    \noindent
    The next three questions kind of build on each other and constitutes a fun
    chain of reasoning!

    \Question
    Argue that in every $n$-node binary search tree, there are exactly $n - 1$
    possible rotations. (CLRS 13.2-2)

    \Question
    Shoe that at most $n - 1$ right rotations suffice to transform any $n$-node
    BST into a right-going chain.

    \Question
    Show that any arbitrary $n$-node binary search tree can be transformed into
    any other arbitrary $n$-node binary search tree using $O(n)$ rotations.
    (CLRS 13.2-4)


\end{Exercise}


\end{document}

