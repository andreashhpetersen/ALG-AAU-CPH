\documentclass{article}

\usepackage[a4paper,margin=3cm,headsep=1cm]{geometry}
\usepackage[utf8]{inputenc}
\usepackage{%
    booktabs,
    fancyhdr,
    pgfgantt,
    titlesec,
    clrscode3e,
    exercise,
    tikz,
    subcaption
}
\usetikzlibrary{trees}

\pagestyle{fancy}

\rhead{Algorithms and Data Structures, S25}
\lhead{Exercises part 2, Lecture 06}
\rfoot{Aalborg University, Copenhagen}
\lfoot{Andreas Holck Høeg-Petersen}

\begin{document}
\thispagestyle{fancy}

\begin{Exercise}[title={Simple training exercises}]

    \Question
    In the last exercise session, you found the red-black tree that results from
    sucessively inserting the keys 41, 38, 31, 12, 19, 8 into an intially empty
    tree. Now show the red-black trees that result from successive deletion of
    the keys in the order 8, 12, 19, 31, 38, 41. (CLRS 13.4-3)
    
    \Question
    Show that if node $y$ in $\proc{RB-Delete}$ is red, then no black-heights
    change.
    
    \Question
    A node $x$ is inserted into a red-black tree with $\proc{RB-Insert}$ and
    then is immediately deleted with $RB-Delete$. Is the resulting red-black
    tree always the same as the initial red-black tree? Justify your answer (for
    example with a minimal example). (CLRS 13.4-8)
    
\end{Exercise}

\begin{Exercise}[title={Programming Task 2}]
    \noindent
    You can spent this time working on the second programming task and ask for
    help, if you want to!

\end{Exercise}


\end{document}


