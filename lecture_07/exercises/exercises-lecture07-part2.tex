\documentclass{article}

\usepackage[a4paper,margin=3cm,headsep=1cm]{geometry}
\usepackage[utf8]{inputenc}
\usepackage{%
    amsmath,
    amssymb,
    booktabs,
    fancyhdr,
    pgfgantt,
    titlesec,
    clrscode3e,
    exercise,
    tikz,
    subcaption
}
\usetikzlibrary{trees}

\pagestyle{fancy}

\rhead{Algorithms and Data Structures, S25}
\lhead{Exercises part 2, Lecture 07}
\rfoot{Aalborg University, Copenhagen}
\lfoot{Andreas Holck Høeg-Petersen}

\begin{document}
\thispagestyle{fancy}

\begin{Exercise}[title={Simple training exercises}]
    \Question
    Consider inserting the keys 10, 22, 31, 4, 15, 28, 17, 88 and 59 into a hash
    table of length $m=11$ using open addressing. Illustrate the result of
    inserting these keys using linear probing with $h(k,i) = (k+i) \mod m$ and
    using double hashing with $h_1(k) = k$ and $h_2(k) = 1 + (k \mod (m-1))$.
    (CLRS 11.4-1)

\end{Exercise}

\begin{Exercise}[title={Exam question from 2024}]
    
    \noindent
    Consider the hash table $T = [4, Nil, Nil, 3, Nil, 18, Nil, Nil]$ with $m =
    8$. Assume zero-indexing.

    \Question
    Insert the keys 7, 5 and 20 using \textit{linear probing} with the auxiliary
    hash function $h'(k) = 3\lfloor k/2 \rfloor$ and show the result.

    \Question
    Insert the keys 7, 5, 20 (in the original table) using \textit{double
    hashing}, with $h_1(k)=k \mod 4$ and $h_2(k)= 2k + 3$ and show the result. 

\end{Exercise}

\begin{Exercise}[title={Fun creative exercises!}]
    Continue with \texttt{hash\_names.py} if you haven't done it yet.
\end{Exercise}


\end{document}

