\documentclass{article}

\usepackage[a4paper,margin=3cm,headsep=1cm]{geometry}
\usepackage[utf8]{inputenc}
\usepackage{%
    amsmath,
    booktabs,
    fancyhdr,
    pgfgantt,
    titlesec,
    clrscode3e,
    exercise,
    tikz,
    subcaption
}
\usetikzlibrary{trees}

\pagestyle{fancy}

\rhead{Algorithms and Data Structures, S25}
\lhead{Exercises part 1, Lecture 07}
\rfoot{Aalborg University, Copenhagen}
\lfoot{Andreas Holck Høeg-Petersen}

\begin{document}
\thispagestyle{fancy}

\begin{Exercise}[title={Simple training exercises}]
    \Question
    Consider a hash table of size $m=1000$ and a corresponding hash function
    $h(k) = \lfloor m(kA - \lfloor kA \rfloor) \rfloor$ for $A = (\sqrt{5} -
    1)/2$. Compute the locations to which the keys 61, 62, 63, 64 and 65 are
    mapped. (CLRS 11.3-4)

    \Question
    Consider a hash table with 9 slots and the hash function $h(k) = k \mod 9$.
    Demonstrate what happens upon inserting the keys 5, 28, 19, 15, 20, 33, 12,
    17, 10 with collisions resolved by chaining. (CLRS 11.2-2)

    \Question
    Professor Marley hypothesizes that he can obtain substantial performance
    gains by modifying the chaining scheme to keep each list in sorted order.
    How does the professor's modification affect the running time for successful
    searches, unsuccessful searches, insertions, and deletions?

\end{Exercise}

\begin{Exercise}[title={Fun creative exercises!}]
    Dive into the python script \texttt{hash\_names.py} on Moodle and try to
    solve the exercises there!
\end{Exercise}


\end{document}

